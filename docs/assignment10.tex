\documentclass[12pt]{article}


\pagestyle{empty}


\begin{document}

\section*{Assignment Set for Laboratory 10: Optional Lab}

EOSC 511/ATSC 506: 

 \begin{enumerate}
\item Modify {\em advection\_funs.py} to solve the following advection
  problem:  The wind is moving along the x-axis with speed \\ $u = 10
  {\rm m\,s}^{-1} \{2 + \cos[2 \pi\,i/(Numpoints-1)]\}$.  The initial
  distribution curve is 290 km in width.  Use your program to
  approximate the curve during 24 hours using the Bott schemes (advection3).

\item Run your program for different orders of approximating polynomials
(up to 4).
Compare the accuracy of approximation for different orders.
Do you see better results with increasing order? Is this true for all
orders from 0 to 4? Is there any peculiarity to odd and even order
polynomials?

\item For odd ordered polynomials, \textit{advection3.m} uses the representation
of $a_{j,k}$ that involves an
extra point to the right of the centre grid point. Modify the table
of coefficients for odd ordered polynomials (\textit{l1\_table} and
\textit{l3\_table}) to use the extra point to the left of the centre
grid point. Run your program again and compare the results of 2 different
representations of  $a_{j,k}$ for order 3. Is one representation better than the other, or about the same, or does each have its own problem? How, do you think
the different representation affects the result?  

\end{enumerate}

\end{document}
