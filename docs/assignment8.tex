\documentclass[12pt]{article}

\pagestyle{empty}

\def\theenumi{\Roman{enumi}}

\begin{document}

\section*{Assignment Set for Laboratory 8: Optional Lab}

EOSC 511/ATSC 506: 

Problem numbers below refer to the problems in the lab text itself.



\begin{enumerate}
\item Do FIRST this subset of problem 3.  Set loop=True, and  compare the efficiency of the 
SOR method (currently implemented, suggest over-relaxation coefficient
of 1.7) and the Jacobi iteration (you need to implement, suggest
coefficient of 1; I find 1.7 unstable).  Also compare to indexing the
loops and doing Jacobi interation by setting loop=False.  You can
time functions using \%timeit

\item Then using the most efficient scheme, 
choose one parameter of the problem
(eg depth, width of ocean, vertical viscosity, wind stress, latitude)
vary it (3 or 4 choices) and compare the solutions.

\item Continue to use the most efficient scheme. Set the wind-stress to zero.
Initialize the stream-function (both psi\_1 and psi\_2) with a blob of fluid somewhere over to the east (say a Gaussian 3/4 of the way across with a radius of several grid points).  Make sure that the boundary condition (psi = 0 on all walls) is enforced.

The blob should move west like a Rossby wave (the natural wave in this problem, a one-way wave, only moves west).  Compare the time-scale of the westward movement to the time-scale of reaching steady state in the wind-driven case.


{\it (Wikipedia entry on Rossby waves is currently okay if you want a quick read about them.  The time scale given for Rossby waves across the ocean is for baroclinic (slow) waves.  You are doing barotropic (fast) waves).}


\end{enumerate}

\end{document}
