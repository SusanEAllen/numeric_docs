\documentclass[12pt]{article}
\usepackage{geometry}
\geometry{letterpaper,top=50pt,hmargin={20mm,20mm},headheight=15pt} 

\input{/home/phil/a409/susan_def}
%\newcommand{\Mathat}{\;{\rm at}\;}
\newcommand{\Mathand}{\;{\rm and}\;}

\newcommand{\pa}{\partial}

\newcommand{\dc}{\pa c}
\newcommand{\de} {\pa \eta}
\newcommand{\deh} {\pa h}
\newcommand{\dphi} {\pa \phi}
\newcommand{\dpres} {\pa p}
\newcommand{\dpsi} {\pa \psi}
\newcommand{\dr}{\pa r}
%\newcommand{\dr}{\pa \rho}
\newcommand{\drho}{\pa \rho}
\newcommand{\ds}{\pa s}
\newcommand{\dS}{\pa S}
\newcommand{\dt}{\pa t}
\newcommand{\dT}{\pa T}
\newcommand{\dTheta}{\pa \Theta}
\newcommand{\du}{\pa u}
\newcommand{\dU}{\pa U}
\newcommand{\dv}{\pa v}
\newcommand{\dV}{\pa V}
\newcommand{\dw}{\pa w}
\newcommand{\dx}{\pa x}
\newcommand{\dxi}{\pa \xi}
\newcommand{\dy}{\pa y}
\newcommand{\dz}{\pa z}
\newcommand{\dzeta}{\pa \zeta}

\newcommand {\ddc} {\frac {\pa}{\dc}}
\newcommand {\ddp} {\frac {\pa}{\dpres}}
\newcommand {\ddr} {\frac {\pa}{\dr}}
\newcommand {\dds} {\frac {\pa}{\ds}}
\newcommand {\ddt} {\frac {\pa}{\dt}}
\newcommand {\DDt} {\frac {D}{Dt}}
\newcommand {\ddx} {\frac {\pa}{\dx}}
\newcommand {\ddy} {\frac {\pa}{\dy}}
\newcommand {\ddz} {\frac {\pa}{\dz}}
\newcommand {\dedt} {\frac {\de}{\dt}}
\newcommand {\dedx} {\frac {\de}{\dx}}
\newcommand {\dedy} {\frac {\de}{\dy}}
\newcommand {\dhdc} {\frac {\deh}{\dc}}
\newcommand {\dhdr} {\frac {\deh}{\dr}}
\newcommand {\dhdt} {\frac {\deh}{\dt}}
\newcommand {\dhdx} {\frac {\deh}{\dx}}
\newcommand {\dhdy} {\frac {\deh}{\dy}}
\newcommand {\dpdc}{\frac {\pa p}{\dc}}
\newcommand {\dpdr}{\frac {\pa p}{\dr}}
\newcommand {\dpds}{\frac {\pa p}{\ds}}
\newcommand {\dpdt}{\frac {\pa p}{\dt}}
\newcommand {\dpdx} {\frac {\pa p}{\dx}}
\newcommand {\dpdy} {\frac {\pa p}{\dy}}
\newcommand {\dpdz} {\frac {\pa p}{\dz}}
\newcommand {\dphidx} {\frac {\dphi}{\dx}}
\newcommand {\dphidy} {\frac {\dphi}{\dy}}
\newcommand {\dphidz} {\frac {\dphi}{\dz}}
\newcommand {\dpsidx} {\frac {\dpsi}{\dx}}
\newcommand {\dpsidy} {\frac {\dpsi}{\dy}}
\newcommand {\dpsidz} {\frac {\dpsi}{\dz}}
\newcommand {\drhodt} {\frac {\pa \rho}{\dt}}
\newcommand {\drhodc} {\frac {\pa \rho}{\dc}}
\newcommand {\drhods} {\frac {\pa \rho}{\ds}}
\newcommand {\drhodx} {\frac {\pa \rho}{\dx}}
\newcommand {\drhody} {\frac {\pa \rho}{\dy}}
\newcommand {\drhodz} {\frac {\pa \rho}{\dz}}
\newcommand {\dSdt}{\frac {\dS}{\dt}}
\newcommand {\dsdz}{\frac {\pa s}{\dz}}
\newcommand {\dTds}{\frac {\dT}{\pa s}}
\newcommand {\dTdt}{\frac {\dT}{\dt}}
\newcommand {\dTdx}{\frac {\dT}{\dx}}
\newcommand {\dTdy}{\frac {\dT}{\dy}}
\newcommand {\dTdz}{\frac {\pa T}{\dz}}
\newcommand {\dThetadt} {\frac {\dTheta}{\dt}}
\newcommand {\dThetadx} {\frac {\dTheta}{\dx}}
\newcommand {\dThetady} {\frac {\dTheta}{\dy}}
\newcommand {\dudt} {\frac {\du}{\dt}}
\newcommand {\dUdt} {\frac {\dU}{\dt}}
\newcommand {\DuDt} {\frac {Du}{Dt}}
\newcommand {\dudc} {\frac {\du}{\dc}}
\newcommand {\duds} {\frac {\du}{\ds}}
\newcommand {\dudx} {\frac {\du}{\dx}}
\newcommand {\dUdx} {\frac {\dU}{\dx}}
\newcommand {\dudy} {\frac {\du}{\dy}}
\newcommand {\dUdy} {\frac {\dU}{\dy}}
\newcommand {\dudz} {\frac {\du}{\dz}}
\newcommand {\dvdt} {\frac {\dv}{\dt}}
\newcommand {\dVdt} {\frac {\dV}{\dt}}
\newcommand {\DvDt} {\frac {Dv}{Dt}}
\newcommand {\dvdc} {\frac {\dv}{\dc}}
\newcommand {\dvdr} {\frac {\dv}{\dr}}
\newcommand {\dvds} {\frac {\dv}{\ds}}
\newcommand {\dvdx} {\frac {\dv}{\dx}}
\newcommand {\dVdx} {\frac {\dV}{\dx}}
\newcommand {\dvdy} {\frac {\dv}{\dy}}
\newcommand {\dVdy} {\frac {\dV}{\dy}}
\newcommand {\dvdz} {\frac {\dv}{\dz}}
\newcommand {\dwdt} {\frac {\dw}{\dt}}
\newcommand {\DwDt} {\frac {Dw}{Dt}}
\newcommand {\dwdx} {\frac {\dw}{\dx}}
\newcommand {\dwdy} {\frac {\dw}{\dy}}
\newcommand {\dwdz} {\frac {\dw}{\dz}}
\newcommand {\dydt} {\frac {\dy}{\dt}}
\newcommand {\dydx} {\frac {\dy}{\dx}}
\newcommand {\dzetadt} {\frac {\pa \zeta}{\dt}}
\newcommand {\dzetadx} {\frac {\pa \zeta}{\dx}}
\newcommand {\dzetady} {\frac {\pa \zeta}{\dy}}
\newcommand {\dzetadz} {\frac {\pa \zeta}{\dz}}
\newcommand {\dzds}{\frac {\dz}{\ds}}
\newcommand {\dzdt} {\frac {\dxi}{\dt}}
\newcommand {\dzdx} {\frac {\dxi}{\dx}}
\newcommand {\dzdy} {\frac {\dxi}{\dy}}


\newcommand{\gp}{g^\prime}
\newcommand{\om}{\omega}
\newcommand{\sgn}{{\rm sgn}}
\newcommand{\sech}{{\rm sech}}

\renewcommand{\deg}[1]{#1^\circ}
\newcommand{\textfrac}[2]{\textstyle \frac {#1} {#2}}
\newcommand{\dotprod}{\cdot}
\newcommand{\bigo}{{\cal O}}
\newcommand{\Kappa}{{\cal K}}
\newcommand{\del}{{\nabla}}
\newcommand{\grad}{{\nabla}}
\newcommand{\cross}{{\times}}
\renewcommand{\div}{{\nabla \cdot}}
\newcommand{\ul}{\underline}
\newcommand{\Chi}{{\cal X}}

\font\sm = cmex10
\font\mbigsm = cmex10 scaled \magstep3
\font\bigsm = cmex10 scaled \magstep4
\font\vbigsm = cmex10 scaled \magstep5

\def\bigsum#1{\hbox{$\textfont3=\vbigsm \displaystyle\sum\limits#1$}}
\def\bigint#1{\hbox{$\textfont3=\bigsm \displaystyle\int#1$}}
\def\medsum#1{\hbox{$\textfont3=\mbigsm \displaystyle\sum\limits#1$}}

\newcommand{\ppt}{$^\circ\!/\!_\circ\!_\circ$}

\begin{document}

\subsection*{Miniproject \# 1 for ATSC 409: Arctic Ocean Near Surface Temperature Maximum}

Consider summer in the Arctic Ocean (Canada Basin).  The surface of
the ocean is partially covered by ice (say ice fraction $\beta$).  The sun
is bright (assume no clouds) and shines most of the day (assume all
day).  Take an incoming radiative flux of 100 W m$^{-2}$, a water albedo of
0.1 and assume that no light penetrates through the ice-covered portion
(i.e. it has lots of snow on top).  Ice is melting, and so the surface
temperature is the freezing temperature of salty water, say
-1$^o$C. Deep in the water column, at 200 m depth, the temperature is
-2$^o$C.  The light energy $I$ decays exponentially with depth with an
e-folding scale of $\alpha$.

In the polar ocean, density is determined by salinity.\footnote{Which
  means its perfectly possible to have colder water above warmer
  water} The surface layer of the ocean of depth ($h$) is well-mixed
and relatively fresh.  Below that is a strongly stratified layer and
then less stratified water.  Assume an eddy-viscosity or mixing
coefficient ($A_h$) of the form

\begin{eqnarray}
 A_{max}, & d < h \\ 
A_{depth} + \left[A_{max}-A_{depth}-A_{dip}(d-h)\right] \exp \left[-0.5(d-h)\right], & d > h.
\end{eqnarray}
where $d$ is the depth, positive in the ocean, measured down from the surface.

An equation for the temperature $T$ as a function of depth, $d$ is
\begin{equation}
\dTdt = \frac {\pa}{\pa d} \left(A_h \frac {\pa T}{\pa d} \right) - \frac 1 {c_p} \frac {\pa I}{\pa d}
\end{equation}
where $c_p = 4000$J kg$^{-1}$ $^o$C$^{-1} = 4 \times 10^6$J m$^{-3}$ $^o$C$^{-1}$ is specific heat. 

Assume steady state and explore the temperature profiles for various ice concentrations $\beta$, mixing profiles(make sure your $A_h$ does not go negative), light attenuation rates ($\alpha$).

Starting parameter suggestions: $\beta = 0.5$, $\alpha = 1/(10$ m), $h = 10$ m, $A_{max} = 1 \times 10^{-2} $m$^2$s$^{-1}$, $A_{depth} = 1 \times 10^{-4} $m$^2$s$^{-1}$, and $A_{dip} = 1.5 \times 10^{-3} $m$^2$s$^{-1}$.  These should give you a Near Surface Temperature Maximum (NSTM) --- see Figure 2 of Jackson et al. https://circle.ubc.ca/handle/2429/34555 but note that the vertical axis is a logscale.

\noindent{\bf Hand-In:}\\
p = on paper\\
e = electronic (email to sallen@eos.ubc.ca)\\
pe = either paper or electronic pdf files
\begin{description}
\item [p] Derivation of the equations you put into your computer model.
\item [p] A paragraph discussing the method of solution.
\item [e] Your code
\item [pe] Results of the base case and your variations (graphs, summary tables)
\item [p] A discussion of the results of your variations
\end{description}

\end{document}
